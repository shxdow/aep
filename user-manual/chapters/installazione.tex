Il progetto è disponibile sia in forma web, all'indirizzo \url{aiticketing.xyz}, oppure sotto forma
di app installabile sul proprio sistema operativo.

Di seguito sono elencati i procedimenti per l'installazione della app sui vari sistemi operativi.

\section{Installazione su macOS}

Per \textit{macOS} è disponibile un file \texttt{.dmg}.
I passi da seguire sono i seguenti:

\begin{enumerate}
  \item Montare il \texttt{.dmg} facendo doppio click su di esso;
  \item Comparirà un disco sul desktop, e dopo averlo aperto si vedrà una finestra come in figura \ref{fig:macos-inst-01};
  \item Trascinare l'icona di \texttt{AI Ticketing} nella cartella delle applicazioni;
  \item La app è ora installata e avviabile, come si vede in figura \ref{fig:macos-inst-02}.
\end{enumerate}

\begin{figure}[H]
  \centering
  \includegraphics[width=0.7\linewidth]{macos-inst-01.jpg}
  \caption{Installazione su macOS (1)}
  \label{fig:macos-inst-01}
\end{figure}

\begin{figure}[H]
  \centering
  \includegraphics[width=0.7\linewidth]{macos-inst-02.jpg}
  \caption{Installazione su macOS (2)}
  \label{fig:macos-inst-02}
\end{figure}

\section{Installazione su Linux}

È disponibile un pacchetto \texttt{.deb} per l'installazione sulle distro Debian-based.
I passi da seguire sono i seguenti:

\begin{enumerate}
  \item Aprire il terminale nella cartella dove si trova il \texttt{.deb};
  \item Eseguire il comando \texttt{sudo apt install ./ticket-frontend\_1.0.0\_amd64.dev};
  \item Inserire, se necessario, la password dell'utente;
  \item Una volta terminato il processo visibile in figura \ref{fig:linux-inst}, la applicazione sarà installata ed utilizzabile.
\end{enumerate}

\begin{figure}[H]
  \centering
  \includegraphics[width=0.8\linewidth]{linux-inst-01.png}
  \caption{Installazione su Linux}
  \label{fig:linux-inst}
\end{figure}

\section{Installazione su Windows}

Per l'installazione su Windows è disponibile un \textit{installer}, che si chiama \texttt{AI Ticketing 1.0.0.exe}.
I passi da seguire sono i seguenti:

\begin{enumerate}
  \item Aprire l'installer, si avvierà una finestra come in figura \ref{fig:win-inst-01};
  \item Una volta terminato l'installer si aprirà la app, come visibile in figura \ref{fig:win-inst-02};
  \item Sul desktop sarà presente il collegamento, e la app è disponibile cercando nel menù start.
\end{enumerate}

\begin{figure}[H]
  \centering
  \includegraphics[width=0.7\linewidth]{win-inst-01.png}
  \caption{Installazione su Windows (1)}
  \label{fig:win-inst-01}
\end{figure}

\begin{figure}[H]
  \centering
  \includegraphics[width=0.7\linewidth]{win-inst-02.png}
  \caption{Installazione su Windows (2)}
  \label{fig:win-inst-02}
\end{figure}