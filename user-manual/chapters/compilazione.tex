\section{Server back-end in Python}

\subsection{Prerequisiti}

È necessario avere \textit{Python 3} e \textit{pip} installati.
Tutti i comandi di seguito saranno effettuati su una shell a propria scelta (o il \textit{Command Prompt} su Windows).

\subsection{Installazione delle dipendenze}

Il codice Python necessita di alcune dipendenze per il corretto funzionamento.
Al fine di isolare il codice ed il suo environment da altri possibilmente esistenti sul sistema
operativo, è necessario attivare un \textit{virtual environment} con il comando
\texttt{python -m venv venv}\footnote{In alcuni casi potrebbe essere necessario usare \texttt{python3}}.

A questo punto, si può attivare il virtual environment con il comando
\begin{itemize}
  \item \texttt{source venv/bin/activate} per i sistemi Unix e Unix-like;
  \item \texttt{venv\textbackslash bin\textbackslash activate} per Windows.
\end{itemize}

Ora è possibile installare le dipendenze, elencate nel file \texttt{requirements.txt},
tramite il comando \texttt{pip install -r requirements.txt}.

\subsection{Esecuzione}

Dopo aver installato le dipendenze ed essersi posti nella cartella \texttt{back-end}, è possibile
eseguire il server.
Una volta attivato il virtual environment, la prima volta bisogna eseguire due
comandi per inizializzare il database ed il superuser:
\begin{itemize}
  \item \texttt{python manage.py migrate}
  \item \texttt{python manage.py createsuperuser}
\end{itemize}

Per avviare il server si usa quindi il comando \texttt{python manage.py runserver},
che di default avvierà un processo in ascolto all'indirizzo \texttt{localhost:8080}.

\subsection{Riassunto}

I comandi da eseguire sono, supponendo di essere inizialmente nella root del progetto
\begin{enumerate}
  \item \texttt{cd back-end}
  \item \texttt{python -m venv venv}
  \item \texttt{source venv/bin/activate}
  \item \texttt{pip install -r requirements.txt}
  \item Solo al primo avvio
    \begin{itemize}
      \item \texttt{python manage.py migrate}
      \item \texttt{python manage.py createsuperuser}
    \end{itemize}
  \item \texttt{python manage.py runserver}
\end{enumerate}

\section{Front-end in JavaScript}

\subsection{Prerequisiti}

È necessario avere \textit{node} e \textit{npm} installati.
Inoltre, seppur non essenziale, è consigliato l'uso di \textit{yarn} per la gestione delle dipendenze.
Tutti i comandi di seguito saranno effettuati su una shell a propria scelta (o il \textit{Command Prompt} su Windows).

Se si vuole usare \texttt{npm} il comando per l'installazione delle dipendenze è \texttt{npm install},
mentre per gli altri comandi basta sostituire \texttt{yarn} con \texttt{npm run}.

\subsection{Installazione delle dipendenze}

Le dipendenze sono gestite tramite \textit{node}.
Per installarle bisogna porsi nella cartella \texttt{front-end} ed eseguire il comando \texttt{yarn install}.


\subsection{Variabili di ambiente}

Per il funzionamento del sito è necessario indicare qual è il server a cui inoltrare le richieste di rete.
Questo viene fatto creando un file \texttt{secrets.js} nella cartella \texttt{src} e ponendoci il seguente contenuto.
\begin{verbatim}
  global.SERVER_ADDRESS = `https://addres.totheserver.com';
\end{verbatim}

\subsection{Esecuzione e compilazione del sito}

Una volta installate le dipendenze, è possibile eseguire i seguenti comandi relativi al sito:
\begin{itemize}
  \item \texttt{yarn start} per avere un server di sviluppo all'indirizzo \texttt{localhost:3000} e che si aggiorna automaticamente ad ogni cambiamento nei file;
  \item \texttt{yarn build} per produrre una versione compilata ed ottimizzata per la produzione
\end{itemize}

\subsection{Esecuzione e compilazione della app}

Il progetto dà anche la possibilità di compilare una app con \textit{Electron}.
Notare che questo metodo è pensato unicamente per la produzione e non per lo sviluppo,
quindi se si desidera modificare gli indirizzi del server essi vanno cambiati sia in \texttt{main.js}
sia in \texttt{electron/menu.js}.

Per effettuare una compilazione con Electron è necessario eseguire i seguenti due comandi in serie:
\texttt{yarn build} e \texttt{yarn electron-build}.

Prestare attenzione alla compilazione con Electron: nel caso il sistema operativo utilizzato non sia macOS,
è possibile che non sia abilitata la compilazione della app per macOS.
In tal caso, basta rimuovere il flag \texttt{-m} dallo script \texttt{electron} nel file \texttt{front-end/package.json}.

Questa compilazione produrrà i relativi eseguibili, installer e pacchetti nella cartella \texttt{dist}.

\subsection{Riassunto}

Di seguito sono riassunti i comandi principali per l'esecuzione e compilazione:
\begin{enumerate}
  \item \texttt{cd front-end}
  \item \texttt{yarn install}
  \item \texttt{echo ``global.SERVER\_ADDRESS = `http://server.address.com''' > src/secrets.js}
  \item Per far partire il server
    \begin{itemize}
      \item \texttt{yarn start}
    \end{itemize}
  \item Per compilare per la produzione
    \begin{itemize}
      \item \texttt{yarn build}
    \end{itemize}
  \item Per compilare la app in Electron
    \begin{itemize}
      \item \texttt{yarn build}
      \item \texttt{yarn electron-build}
    \end{itemize}
\end{enumerate}