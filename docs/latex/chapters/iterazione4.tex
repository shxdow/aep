\section{Continuous Integration}

A questo punto dello sviluppo, avendo creato un MVP funzionante, è stato deciso di apportare miglioramenti
all'aspetto infrastrutturale, utilizzando le idee del paradigma \textit{Continuos Integration}.

Tale miglioramente è stato implementato utilizzando \textit{Travis}, un software in grado di eseguire script
e segnalare lo stato di terminazione di tali istruzioni. Essi possono essere utilizzati per automatizzare
l'esecuzione dei test ogni qualvolta viene aggiunto del codice al repository git di riferimento.

In questo modo abbiamo semplificato di molto il ciclo di sviluppo e modifica del codice, garantendo di avere
sempre l'esecuzione dei test senza doverli avviare manualmente.

\section{Electron}

Electron è una libreria JavaScript utilizzata per la creazione di applicazioni cross-platform
tramite i linguaggi che si trovano tipicamente in ambito Web, ossia JavaScript, HTML e CSS.
È usata in molti progetti, come ad esempio \textit{Visual Studio Code}, \textit{Twitch} desktop,
\textit{Microsoft Teams}, \textit{Postman} e svariate altre\footnote{\url{https://www.electronjs.org/apps}}.

\subsection{Integrazione con il front-end}

L'integrazione con il codice front-end è stata relativamente facile.
È stato creato un file (\texttt{main.js}) che si occupasse della gestione delle finestre dell'applicazione
tramite le API di Electron.

Inoltre, è necessario indicare quali file considerare nella creazione dei vari \textit{artifact} prodotti da
Electron: abbiamo deciso di includere i file già compressi e ottimizzati per il web, in modo da accelerare
le performance della applicazione.

Infine, sono state anche presi in considerazione i diversi comportamenti standard che i principali sistemi
operativi (macOS, Linux e Windows) definiscono per i propri programmi.
Ad esempio, se il progetto viene compilato con Electron per macOS, allora la chiusura della finestra non
determinerà la chiusura dell'applicazione, a differenza di quanto accade se compilato per Linux o Windows,
come si può vedere nel seguente estratto del file \texttt{main.js}\footnote{Il sistema macOS è identificato dalla piattaforma `darwin'}.

\begin{lstlisting}
app.whenReady().then(createWindow);

app.on('window-all-closed', () => {
  if (process.platform !== 'darwin') {
    app.quit();
  }
});

app.on('activate', () => {
  if (BrowserWindow.getAllWindows().length === 0) {
    createWindow();
  }
});
\end{lstlisting}

In ogni caso, oltre agli eseguibili ed alle dovute risorse, vengono compilati anche i vari installer
per ogni piattaforma, così da garantire una maggiore facilità di installazione e condivisione.

\subsection{Gestione dei cookies}

L'unico problema incontrano nell'integrazione con Electron è stato nella gestione dei cookies.
Per motivi di sicurezza, i cookie sicuri non vengono salvati sul protocollo \texttt{file}, che è quello
usato da Electron per la gestione dei file HTML.

Quindi, per garantire il corretto funzionamento, è stato implementato un piccolo modulo che si occupa di
intercettare le richieste di rete fatte da e verso il nostro server e di aggiungere in modo trasparente
i cookie scambiati.

Questi cookie esistono solamente in memoria\footnote{Sono quindi persi alla chiusura dell'applicazione}
e non sono mai inviati a domini diversi da quello a cui risiede il back-end, garantendo così un'adeguata
sicurezza.
