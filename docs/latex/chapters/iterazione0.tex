\section{Introduzione}

\section{Specifiche}
\subsection{Casi d'uso}

\subsubsection{Attori}
Gli attori che devono utilizzare il sistema sono:

\begin{itemize}
  \item \textit{Utente registrato}: un utente esterno all'azienda che si è registrato al portale per la gestione dei ticket;
  \item \textit{Utente non registrato}: un qualsiasi utente esterno all'azienda che non si è registrato al portale;
  \item \textit{Operatore}: un dipendente dell'azienda registrato nel sistema di gestione dei ticket;
  \item \textit{Amministratore}: un dipendente dell'azienda registrato, con compiti di gestione.
\end{itemize}

\subsubsection{Elenco dei casi d'uso}
Nella Tabella \ref{tab:casi-uso} vengono descritti i casi d'uso individuati con i rispettivi codici, mentre il diagramma UML dei casi d'uso è in Figura \ref{fig:diag-use-cases}.

\begin{table}
  \centering
  \begin{tabular}{l|m{12cm}|}
    \hline
    Codice & Descrizione \\
    \hline
    \hline \texttt{U1} & Un utente registrato crea un nuovo ticket da form                                                                      \\
    \hline \texttt{U2} & Un utente non registrato crea un nuovo ticket inviando una mail                                                        \\
    \hline \texttt{U3} & Un utente (sia registrato sia non registrato) può controllare lo stato del ticket                                      \\
    \hline \texttt{U4} & Un utente registrato può usare il portale per comunicare (inserendo dei commenti) con l'operatore assegnato al ticket  \\
    \hline \texttt{U5} & Un utente crea e gestisce l'account                                                                                    \\
    \hline \texttt{S1} & Il sistema assegna in automatico un nuovo ticket ad un gruppo di operatori o ad un operatore specifico                 \\
    \hline \texttt{S2} & Il sistema fornisce una mail ad un utente non registrato per poter comunicare con l'operatore                          \\
    \hline \texttt{O1} & Un operatore può cambiare lo stato del ticket (aperto/chiuso/assegnato/in corso)                                       \\
    \hline \texttt{O2} & Un operatore può inserire commenti sui ticket a cui è assegnato                                                        \\
    \hline \texttt{A1.1} & Un amministratore interno all'azienda inserisce un nuovo operatore                                                   \\
    \hline \texttt{A1.2} & Un amministratore interno all'azienda modifica un operatore                                                          \\
    \hline \texttt{A1.3} & Un amministratore interno all'azienda elimina un operatore                                                           \\
    \hline \texttt{A2} & Un amministratore interno all'azienda assegna manualmente i ticket ad un operatore specifico                           \\
    \hline \texttt{G1} & Caso d'uso generico che rappresenta l'inserimento di un ticket                                                         \\
    \hline
  \end{tabular}
  \caption{Casi d'uso}
  \label{tab:casi-uso}
\end{table}


\begin{figure}[H]
  \centering
  \includegraphics[width=\linewidth]{use-cases-uml.png}
  \caption{Diagramma UML dei casi d'uso}
  \label{fig:diag-use-cases}
\end{figure}

\subsubsection{Dettaglio dei casi d'uso}
Nel seguente elenco puntanto, verranno analizzati in maniera più dettagliata tutti i casi d'uso, identificandone descrizione, attori, precondizioni, passi principali, situazioni
eccezionali e postcondizioni. 

\begin{itemize}
  \item \textbf{U1}: \textit{Inserimento ticket via form}
    \begin{itemize}
      \item \textbf{Descrizione}: l'utente desidera creare un nuovo ticket tramite form web;
      \item \textbf{Attori}: utente;
      \item \textbf{Precondizioni}: l'utente ha effettuato il login e si trova nel form di creazione ticket;
      \item \textbf{Passi principali}: 
        \begin{enumerate}
          \item L'utente inserisce titolo e descrizione del ticket;
          \item L'utente crea il nuovo ticket;
          \item Il sistema inserisce il nuovo ticket in database;
          \item Il sistema assegna in automatico il ticket (S1).
        \end{enumerate}
      \item \textbf{Situazioni eccezionali}: nessuna;
      \item \textbf{Postcondizioni}: il ticket è inserito in un database.
    \end{itemize}
  \item \textbf{U2}: \textit{Inserimento ticket via mail}
    \begin{itemize}
      \item \textbf{Descrizione}: l'utente desidera creare un nuovo ticket tramite email;
      \item \textbf{Attori}: utente;
      \item \textbf{Precondizioni}: nessuna;
      \item \textbf{Passi principali}: 
        \begin{enumerate}
          \item L'utente invia una mail al sistema di ticketing;
          \item Il sistema inserisce i dati in database;
          \item Il sistema genera una mail di comunicazione (S2);
          \item L'utente viene notificato dal sistema dell'avvenuta ricezione.
        \end{enumerate}
      \item \textbf{Situazioni eccezionali}: nessuna;
      \item \textbf{Postcondizioni}: il ticket è inserito in un database.
    \end{itemize}
  \item \textbf{U3}: \textit{Controllo stato ticket}
    \begin{itemize}
      \item \textbf{Descrizione}: l'utente controlla lo stato del ticket (aperto/assegnato/in corso/chiuso);
      \item \textbf{Attori}: utente;
      \item \textbf{Precondizioni}: l'utente conosce l'identificativo del ticket;
      \item \textbf{Passi principali}: 
        \begin{enumerate}
          \item L'utente naviga alla pagina di stato dello specifico ticket;
          \item Il sistema invia all'utente le varie informazioni sul ticket, tra cui titolo, descrizione e stato;
        \end{enumerate}
      \item \textbf{Situazioni eccezionali}: se il ticket è stato inviato tramite mail, è visibile solamente lo stato del ticket;
      \item \textbf{Postcondizioni}: nessuna.
    \end{itemize}
  \item \textbf{U4}: \textit{Inserimento di commenti sul ticket}
    \begin{itemize}
      \item \textbf{Descrizione}: l'utente inserisce un commento in un ticket già aperto;
      \item \textbf{Attori}: utente;
      \item \textbf{Precondizioni}: l'utente deve aver effettuato il login ed il ticket deve esistere;
      \item \textbf{Passi principali}: 
        \begin{enumerate}
          \item L'utente inserisce un nuovo commento;
          \item L'utente salva il commento;
          \item Il sistema inserisce il commento in database.
        \end{enumerate}
      \item \textbf{Situazioni eccezionali}: se il ticket è assegnato ad un operatore, l'operatore viene notificato;
      \item \textbf{Postcondizioni}: il commento è inserito in database.
    \end{itemize}
  \item \textbf{U5}: \textit{Gestione account}
    \begin{itemize}
      \item \textbf{Descrizione}: l'utente modifica i dati inseriti in fase di registrazione;
      \item \textbf{Attori}: utente;
      \item \textbf{Precondizioni}: l'utente deve aver effettuato il login;
      \item \textbf{Passi principali}: 
        \begin{enumerate}
          \item L'utente inserisce i nuovi dati da aggiornare;
          \item L'utente salva i nuovi dati;
          \item Il sistema riceve i dati e aggiorna le informazioni in database.
        \end{enumerate}
      \item \textbf{Situazioni eccezionali}: nessuna;
      \item \textbf{Postcondizioni}: il ticket viene inserito in database dal sistema.
    \end{itemize}
  \item \textbf{S1}: \textit{Assegnamento automatico del ticket}
    \begin{itemize}
      \item \textbf{Descrizione}: il ticket è assegnato automaticamente ad un gruppo;
      \item \textbf{Attori}: nessuno;
      \item \textbf{Precondizioni}: il sistema ha ricevuto e inserito in database un nuovo ticket;
      \item \textbf{Passi principali}: 
        \begin{enumerate}
          \item Il sistema valuta il miglior gruppo a cui assegnare il ticket sulla base delle informazioni presenti in database;
          \item Il sistema aggiorna le informazioni in database di conseguenza;
          \item Il sistema notifica il corrispondente gruppo di operatori.
        \end{enumerate}
      \item \textbf{Situazioni eccezionali}: il ticket può non essere assegnato se non sono presenti abbastanza informazioni;
      \item \textbf{Postcondizioni}: il ticket è assegnato ad un gruppo di operatori.
    \end{itemize}
  \item \textbf{S2}: \textit{Generazione mail di comunicazione}
    \begin{itemize}
      \item \textbf{Descrizione}: il sistema genera una mail che comunica all'utente per la comunicazione con gli operatori;
      \item \textbf{Attori}: nessuno;
      \item \textbf{Precondizioni}: il sistema ha ricevuto e inserito in database un nuovo ticket;
      \item \textbf{Passi principali}: 
        \begin{enumerate}
          \item Il sistema genera un indirizzo mail per le comunicazioni sullo specifico ticket;
          \item Il sistema comunica all'utente questo indirizzo mail.
        \end{enumerate}
      \item \textbf{Situazioni eccezionali}: nessuna;
      \item \textbf{Postcondizioni}: l'utente è in possesso di una mail per la comunicazione con gli operatori.
    \end{itemize}
  \item \textbf{O1}: \textit{Gestione stato ticket}
    \begin{itemize}
      \item \textbf{Descrizione}: l'operatore cambia lo stato del ticket;
      \item \textbf{Attori}: operatore;
      \item \textbf{Precondizioni}: il ticket deve esistere in database e dev'essere stato assegnato, l'operatore deve aver effettuato il login;
      \item \textbf{Passi principali}: 
        \begin{enumerate}
          \item L'operatore modifica lo stato del ticket;
          \item Il sistema aggiorna lo stato del ticket in database;
          \item Il sistema notifica l'utente dell'avvenuto cambiamento di stato.
        \end{enumerate}
      \item \textbf{Situazioni eccezionali}: nessuna;
      \item \textbf{Postcondizioni}: il ticket ha il nuovo stato e l'utente è stato avvisato.
    \end{itemize}
  \item \textbf{O2}: \textit{Inserimento di commenti sul ticket}
    \begin{itemize}
      \item \textbf{Descrizione}: un operatore aggiunge commenti su uno specifico ticket;
      \item \textbf{Attori}: operatore;
      \item \textbf{Precondizioni}: il ticket deve esistere in database e dev'essere stato assegnato, l'operatore deve aver effettuato il login;
      \item \textbf{Passi principali}: 
        \begin{enumerate}
          \item L'operatore inserisce un nuovo commento;
          \item L'operatore salva il commento;
          \item Il sistema inserisce il commento in database;
          \item Il sistema notifica l'utente dell'avvenuto inserimento del commento.
        \end{enumerate}
      \item \textbf{Situazioni eccezionali}: nessuna;
      \item \textbf{Postcondizioni}: il commento è inserito in database.
    \end{itemize}
  \item \textbf{A1.1}: \textit{Gestione operatori - inserimento}
    \begin{itemize}
      \item \textbf{Descrizione}: un amministratore inserisce un nuovo operatore;
      \item \textbf{Attori}: amministratore;
      \item \textbf{Precondizioni}: l'amministratore deve aver effettuato il login;
      \item \textbf{Passi principali}: 
        \begin{enumerate}
          \item L'amministratore inserisce i dati del nuovo operatore;
          \item L'amministratore salva i nuovi dati;
          \item Il sistema inserisce i dati in database.
        \end{enumerate}
      \item \textbf{Situazioni eccezionali}: nessuna;
      \item \textbf{Postcondizioni}: il nuovo operatore risulta inserito.
    \end{itemize}
  \item \textbf{A1.2}: \textit{Gestione operatori - modifica}
    \begin{itemize}
      \item \textbf{Descrizione}: un amministratore modifica un operatore;
      \item \textbf{Attori}: amministratore;
      \item \textbf{Precondizioni}: l'amministratore deve aver effettuato il login;
      \item \textbf{Passi principali}: 
        \begin{enumerate}
          \item L'amministratore modifica i dati dell'operatore;
          \item L'amministratore salva i nuovi dati;
          \item Il sistema modifica i dati in database.
        \end{enumerate}
      \item \textbf{Situazioni eccezionali}: nessuna;
      \item \textbf{Postcondizioni}: l'operatore risulta aggiornato.
    \end{itemize}
  \item \textbf{A1.3}: \textit{Gestione operatori - elimina}
    \begin{itemize}
      \item \textbf{Descrizione}: un amministratore elimina un operatore;
      \item \textbf{Attori}: amministratore;
      \item \textbf{Precondizioni}: l'amministratore deve aver effettuato il login;
      \item \textbf{Passi principali}: 
        \begin{enumerate}
          \item L'amministratore elimina un operatore;
          \item Il sistema aggiorna i dati in database segnando l'operatore come "eliminato".
        \end{enumerate}
      \item \textbf{Situazioni eccezionali}: nessuna;
      \item \textbf{Postcondizioni}: l'operatore risulta eliminato.
    \end{itemize}
  \item \textbf{A2}: \textit{Assegnamento manuale del ticket}
    \begin{itemize}
      \item \textbf{Descrizione}: un amministratore assegna manualmente il ticket ad un gruppo di operatori;
      \item \textbf{Attori}: amministratore;
      \item \textbf{Precondizioni}: l'amministratore deve aver effettuato il login;
      \item \textbf{Passi principali}: 
        \begin{enumerate}
          \item L'amministratore cambia i dati del ticket inserendo il gruppo a cui assegnarlo;
          \item L'amministratore salva i dati;
          \item Il sistema modifica i dati in database
        \end{enumerate}
      \item \textbf{Situazioni eccezionali}: nessuna;
      \item \textbf{Postcondizioni}: il ticket risulta assegnato al gruppo specificato.
    \end{itemize}
  \item \textbf{G1}: \textit{Assegnamento del ticket}
    \begin{itemize}
      \item \textbf{Descrizione}: caso d'uso generico che rappresenta l'assegnamento di un ticket ad un gruppo di operatori;
      \item \textbf{Attori}: nessuno;
      \item \textbf{Precondizioni}: nessuna;
      \item \textbf{Passi principali}: nessuno;
      \item \textbf{Situazioni eccezionali}: nessuna;
      \item \textbf{Postcondizioni}: il ticket risulta assegnato ad un gruppo di operatori.
    \end{itemize}
\end{itemize}




\subsection{Requisiti funzionali}

Di seguito è presente l'elenco completo di tutte le specifiche funzionali necessarie
al fine della realizzazione del software:

\begin{enumerate}
  \item \textit{Creazione di un account}, ovvero l'inserimento dei dati relativi per la creazione di un account utente (tra cui nome, password ed email);
  \item \textit{Gestione dell'account}, ovvero la modifica delle informazioni dell'account utente;
  \item \textit{Login} e \textit{logout} di utenti ed operatori;
  \item \textit{Creazione di ticket tramite form} da parte di un utente autenticato;
  \item \textit{Creazione di ticket tramite mail} da parte del sistema;
  \item \textit{Controllo dello stato del ticket}, sia con informazioni aggiuntive (per utenti autenticati) sia senza di esse (per utenti non autenticati);
  \item \textit{Assegnamento automatico del ticket} da parte del sistema ad un gruppo di operatori;
  \item \textit{Assegnamento manuale del ticket} da parte dell'amministratore ad un gruppo di operatori;
  \item \textit{Creazione, modifica ed eliminazione di operatori} da parte dell'amministratore.
\end{enumerate}

Tutte queste funzionalità sono visibili riassunte nella Tabella \ref{tab:requisiti-funzionali}.

\begin{table}
  \centering
  \begin{tabular}{|l|l|}
    \hline
    Codice & Descrizione \\
    \hline
    \hline \texttt{FU1} & Creazione account             \\
    \hline \texttt{FU2} & Gestione account              \\
    \hline \texttt{FT1} & Creazione ticket tramite form \\
    \hline \texttt{FT2} & Creazione ticket tramite mail \\
    \hline \texttt{FT3} & Controllo stato ticket        \\
    \hline \texttt{FT4} & Inserimento commenti          \\
    \hline \texttt{FT5} & Gestione stato ticket         \\
    \hline \texttt{FT6} & Assegnameto automatico ticket \\
    \hline \texttt{FT7} & Assegnameto manuale ticket    \\
    \hline \texttt{FO1} & Creazione operatore           \\
    \hline \texttt{FO2} & Modifica operatore            \\
    \hline \texttt{FO3} & Eliminazione operatore        \\
    \hline
  \end{tabular}
  \caption{Requisiti funzionali}
  \label{tab:requisiti-funzionali}
\end{table}

\section{Architettura software}

\subsection{Component Diagram}

\begin{figure}[H]
  \centering
  \includegraphics[width=\linewidth]{component-diagram.png}
  \caption{Diagramma dei componenti}
  \label{fig:diag-componenti}
\end{figure}

\subsection{Class Diagram}

\begin{figure}[H]
  \centering
  \includegraphics[width=\linewidth]{class-diagram.png}
  \caption{Diagramma delle classi}
  \label{fig:diag-class}
\end{figure}

\section{Architettura hardware}
\subsection{Architettura hardware}

\begin{figure}[H]
  \centering
  \includegraphics[width=\linewidth]{architecture.png}
  \caption{Architettura hardware}
  \label{fig:diag-arch-hw}
\end{figure}

\subsection{Deployment Diagram}

\begin{figure}[H]
  \centering
  \includegraphics[width=\linewidth]{deployment-diagram.png}
  \caption{Diagramma di deployment}
  \label{fig:diag-deployment}
\end{figure}