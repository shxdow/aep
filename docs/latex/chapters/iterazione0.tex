\section{Introduzione}

\section{Specifiche}
\subsection{Casi d'uso}

TODO: Elenco dei casi d'uso

\begin{figure}[H]
  \centering
  \includegraphics[width=\linewidth]{use-cases-uml.png}
  \caption{Diagramma UML dei casi d'usi}
  \label{fig:diag-use-cases}
\end{figure}

\subsection{Requisiti funzionali}

Di seguito è presente l'elenco completo di tutte le specifiche funzionali necessarie
al fine della realizzazione del software:

\begin{enumerate}
  \item \textit{Creazione di un account}, ovvero l'inserimento dei dati relativi per la creazione di un account utente (tra cui nome, password ed email);
  \item \textit{Gestione dell'account}, ovvero la modifica delle informazioni dell'account utente;
  \item \textit{Login} e \textit{logout} di utenti ed operatori;
  \item \textit{Creazione di ticket tramite form} da parte di un utente autenticato;
  \item \textit{Creazione di ticket tramite mail} da parte del sistema;
  \item \textit{Controllo dello stato del ticket}, sia con informazioni aggiuntive (per utenti autenticati) sia senza di esse (per utenti non autenticati);
  \item \textit{Assegnamento automatico del ticket} da parte del sistema ad un gruppo di operatori;
  \item \textit{Assegnamento manuale del ticket} da parte dell'amministratore ad un gruppo di operatori;
  \item \textit{Creazione, modifica ed eliminazione di operatori} da parte dell'amministratore.
\end{enumerate}

Tutte queste funzionalità sono visibili riassunte nella tabella \ref{tab:requisiti-funzionali}.

\begin{table}
  \centering
  \begin{tabular}{|l|l|}
    \hline
    Codice & Descrizione \\
    \hline
    \hline \texttt{FU1} & Creazione account             \\
    \hline \texttt{FU2} & Gestione account              \\
    \hline \texttt{FT1} & Creazione ticket tramite form \\
    \hline \texttt{FT2} & Creazione ticket tramite mail \\
    \hline \texttt{FT3} & Controllo stato ticket        \\
    \hline \texttt{FT4} & Inserimento commenti          \\
    \hline \texttt{FT5} & Gestione stato ticket         \\
    \hline \texttt{FT6} & Assegnameto automatico ticket \\
    \hline \texttt{FT7} & Assegnameto manuale ticket    \\
    \hline \texttt{FO1} & Creazione operatore           \\
    \hline \texttt{FO2} & Modifica operatore            \\
    \hline \texttt{FO3} & Eliminazione operatore        \\
    \hline
  \end{tabular}
  \caption{Requisiti funzionali}
  \label{tab:requisiti-funzionali}
\end{table}

\section{Architettura software}

\subsection{Component Diagram}

\begin{figure}[H]
  \centering
  \includegraphics[width=\linewidth]{component-diagram.png}
  \caption{Diagramma dei componenti}
  \label{fig:diag-componenti}
\end{figure}

\subsection{Class Diagram}

\begin{figure}[H]
  \centering
  \includegraphics[width=\linewidth]{class-diagram.png}
  \caption{Diagramma delle classi}
  \label{fig:diag-class}
\end{figure}

\section{Architettura hardware}
\subsection{Architettura hardware}

\begin{figure}[H]
  \centering
  \includegraphics[width=\linewidth]{architecture.png}
  \caption{Architettura hardware}
  \label{fig:diag-arch-hw}
\end{figure}

\subsection{Deployment Diagram}

\begin{figure}[H]
  \centering
  \includegraphics[width=\linewidth]{deployment-diagram.png}
  \caption{Diagramma di deployment}
  \label{fig:diag-deployment}
\end{figure}