\section{Introduzione}

In ambito aziendale è solito avere un sistema cosiddetto di \textit{ticketing} per il tracciamento e la risoluzione di problemi,
principalmente relativi ai prodotti software venduti, tra i dipendenti dell'azienda stessa e gli utenti utilizzatori.
Questa gestione consente di tenere traccia dell'evoluzione della comunicazione e dei vari passi risolutivi che portano alla
chiusura del ticket ed alla sistemazione di eventuali bug e regressioni.

Il sistema di \textit{ticketing} da noi realizzato implementa tutte le funzionalità comunemente presenti negli altri sistemi,
ma si differenzia per via dell'assegnamento automatico dei ticket aperti a varie unità organizzative per un miglior filtraggio
ed una migliore organizzazione di informazioni.
Questo assegnamento automatico è effettuato tramite tecniche di classificazione non supervisionate, come ad esempio \textit{k-means},
che analizzano solamente i dati che definiscono un ticket.

\section{Analisi dei requisiti}

\subsection{Analisi del contesto}

La gestione dei ticket, seppur necessaria e fondamentale per un'azienda, può rappresentare
un costo in termini sia di risorse sia di tempo, in quanto il ticket dev'essere letto, compreso e
smistato per poterlo indirizzare alla persona (o gruppo di persone) più in grado di risolvere il problema.

Il processo di evasione dei ticket rappresenta dunque un costo ulteriore che può diminuire l'efficienza
dei processi aziendali.

\subsection{Studio di fattibilità}

Un sistema per l'assegnamento automatico che si basa su metodi non supervisionati può ridurre i costi definiti alla sezione precedente.
Per realizzare questa funzionalità bisogna quindi implementare un algoritmo di classificazione che, estraendo opportune
informazioni dal contenuto di un ticket, provvede alla sua classificazione in più gruppi definiti dall'azienda stessa.

I costi principali relativi alla soluzione sono legati allo sviluppo del software
e alla manutenzione del server per l'\textit{hosting} del sistema.

\section{Specifiche}

In questa sezione sono presenti le varie specifiche che definiscono la struttura ed il comportamento del sistema.

\subsection{Casi d'uso}

Nelle seguenti sezioni sono analizzati gli attori ed i casi d'uso che definiscono il comportamento del sistema.

\subsubsection{Attori}
Gli attori che devono utilizzare il sistema sono:

\begin{itemize}
  \item \textit{Utente registrato}, un utente esterno all'azienda che si è registrato al portale per la gestione dei ticket;
  \item \textit{Utente non registrato}, un qualsiasi utente esterno all'azienda che non si è registrato al portale;
  \item \textit{Operatore}, un dipendente dell'azienda;
  \item \textit{Amministratore}, un dipendente dell'azienda con compiti di gestione.
\end{itemize}

\subsubsection{Elenco dei casi d'uso}
Nella tabella \ref{tab:casi-uso} vengono descritti i casi d'uso individuati con i rispettivi codici,
mentre il diagramma UML dei casi d'uso è visibile in figura \ref{fig:diag-use-cases}.

\begin{table}
  \centering
  \begin{tabular}{|l|m{12cm}|}
    \hline
    Codice & Descrizione \\
    \hline
    \hline \texttt{U1} & Un utente registrato crea un nuovo ticket da form \\
    \hline \texttt{U2} & Un utente non registrato crea un nuovo ticket inviando una mail \\
    \hline \texttt{U3} & Un utente (sia registrato sia non registrato) può controllare lo stato del ticket \\
    \hline \texttt{U4} & Un utente registrato può usare il portale per comunicare (inserendo dei commenti) con l'operatore assegnato al ticket \\
    \hline \texttt{U5} & Un utente crea e gestisce l'account \\
    \hline \texttt{S1} & Il sistema assegna in automatico un nuovo ticket ad un gruppo di operatori o ad un operatore specifico \\
    \hline \texttt{S2} & Il sistema fornisce una mail ad un utente non registrato per poter comunicare con l'operatore \\
    \hline \texttt{O1} & Un operatore può cambiare lo stato del ticket (aperto, chiuso, assegnato, in corso \\
    \hline \texttt{O2} & Un operatore può inserire commenti sui ticket a cui è assegnato \\
    \hline \texttt{A1.1} & Un amministratore interno all'azienda inserisce un nuovo operatore \\
    \hline \texttt{A1.2} & Un amministratore interno all'azienda modifica un operatore \\
    \hline \texttt{A1.3} & Un amministratore interno all'azienda elimina un operatore \\
    \hline \texttt{A2} & Un amministratore interno all'azienda assegna manualmente i ticket ad un operatore specifico \\
    \hline \texttt{G1} & Caso d'uso generico che rappresenta l'inserimento di un ticket \\
    \hline
  \end{tabular}
  \caption{Casi d'uso}
  \label{tab:casi-uso}
\end{table}

\begin{figure}[H]
  \centering
  \includegraphics[width=0.85\linewidth]{use-cases-uml.png}
  \caption{Diagramma UML dei casi d'uso}
  \label{fig:diag-use-cases}
\end{figure}

\subsubsection{Dettaglio dei casi d'uso}
Nel seguente elenco sono analizzati in maniera dettagliata tutti i casi d'uso, identificandone descrizione, attori, precondizioni, passi principali, situazioni
eccezionali e postcondizioni. 

\begin{itemize}
  \item \textbf{U1}: \textit{Creazione ticket via form}
    \begin{itemize}
      \item \textbf{Descrizione}: l'utente desidera creare un nuovo ticket tramite form
      \item \textbf{Attori}: utente
      \item \textbf{Precondizioni}: l'utente ha effettuato il login e si trova nel form di creazione ticket
      \item \textbf{Passi principali}: 
        \begin{enumerate}
          \item L'utente inserisce titolo e descrizione del ticket
          \item L'utente crea il nuovo ticket
          \item Il sistema inserisce il nuovo ticket in database
          \item Il sistema assegna in automatico il ticket
        \end{enumerate}
      \item \textbf{Situazioni eccezionali}: nessuna
      \item \textbf{Postcondizioni}: il ticket è inserito in un database
    \end{itemize}
  \item \textbf{U2}: \textit{Creazione ticket via mail}
    \begin{itemize}
      \item \textbf{Descrizione}: l'utente desidera creare un nuovo ticket tramite email
      \item \textbf{Attori}: utente
      \item \textbf{Precondizioni}: nessuna
      \item \textbf{Passi principali}: 
        \begin{enumerate}
          \item L'utente invia una mail al sistema di ticketing
          \item Il sistema inserisce i dati in database
          \item Il sistema genera una mail di comunicazione
          \item L'utente viene notificato dal sistema dell'avvenuta ricezione
        \end{enumerate}
      \item \textbf{Situazioni eccezionali}: nessuna
      \item \textbf{Postcondizioni}: il ticket è inserito in un database
    \end{itemize}
  \item \textbf{U3}: \textit{Controllo stato ticket}
    \begin{itemize}
      \item \textbf{Descrizione}: l'utente controlla lo stato del ticket (aperto, assegnato, in corso, chiuso)
      \item \textbf{Attori}: utente
      \item \textbf{Precondizioni}: l'utente conosce l'identificativo del ticket
      \item \textbf{Passi principali}: 
        \begin{enumerate}
          \item L'utente naviga alla pagina di stato dello specifico ticket
          \item Il sistema invia all'utente le varie informazioni sul ticket, tra cui titolo, descrizione e stato
        \end{enumerate}
      \item \textbf{Situazioni eccezionali}: se il ticket è stato creato tramite mail, è visibile solamente lo stato del ticket
      \item \textbf{Postcondizioni}: nessuna
    \end{itemize}
  \item \textbf{U4}: \textit{Inserimento di commenti sul ticket}
    \begin{itemize}
      \item \textbf{Descrizione}: l'utente inserisce un commento in un ticket già aperto
      \item \textbf{Attori}: utente
      \item \textbf{Precondizioni}: l'utente deve aver effettuato il login ed il ticket deve esistere
      \item \textbf{Passi principali}: 
        \begin{enumerate}
          \item L'utente inserisce un nuovo commento sul ticket desiderato
          \item L'utente salva il commento
          \item Il sistema inserisce il commento in database
        \end{enumerate}
      \item \textbf{Situazioni eccezionali}: nessuna
      \item \textbf{Postcondizioni}: il commento è inserito in database
    \end{itemize}
  \item \textbf{U5}: \textit{Gestione account}
    \begin{itemize}
      \item \textbf{Descrizione}: l'utente modifica i dati relativi al suo account inseriti in fase di registrazione
      \item \textbf{Attori}: utente
      \item \textbf{Precondizioni}: l'utente deve aver effettuato il login
      \item \textbf{Passi principali}:
        \begin{enumerate}
          \item L'utente inserisce i nuovi dati da aggiornare
          \item L'utente salva i nuovi dati
          \item Il sistema riceve i dati e aggiorna le informazioni in database
        \end{enumerate}
      \item \textbf{Situazioni eccezionali}: nessuna
      \item \textbf{Postcondizioni}: il ticket viene inserito in database dal sistema
    \end{itemize}
  \item \textbf{S1}: \textit{Assegnamento automatico del ticket}
    \begin{itemize}
      \item \textbf{Descrizione}: il ticket è assegnato automaticamente ad un gruppo
      \item \textbf{Attori}: nessuno
      \item \textbf{Precondizioni}: il sistema ha ricevuto e inserito in database un nuovo ticket
      \item \textbf{Passi principali}: 
        \begin{enumerate}
          \item Il sistema valuta il miglior gruppo a cui assegnare il ticket sulla base delle informazioni presenti in database
          \item Il sistema aggiorna le informazioni in database di conseguenza
          \item Il sistema notifica il corrispondente gruppo di operatori
        \end{enumerate}
      \item \textbf{Situazioni eccezionali}: il ticket può non essere assegnato se non sono presenti abbastanza informazioni
      \item \textbf{Postcondizioni}: il ticket è assegnato ad un gruppo di operatori
    \end{itemize}
  \item \textbf{S2}: \textit{Generazione mail di comunicazione}
    \begin{itemize}
      \item \textbf{Descrizione}: il sistema genera una mail che comunica all'utente per la comunicazione con gli operatori
      \item \textbf{Attori}: nessuno
      \item \textbf{Precondizioni}: il sistema ha ricevuto e inserito in database un nuovo ticket
      \item \textbf{Passi principali}: 
        \begin{enumerate}
          \item Il sistema genera un indirizzo mail per le comunicazioni sullo specifico ticket
          \item Il sistema comunica all'utente questo indirizzo mail
        \end{enumerate}
      \item \textbf{Situazioni eccezionali}: nessuna
      \item \textbf{Postcondizioni}: l'utente è in possesso di una mail per la comunicazione con gli operatori
    \end{itemize}
  \item \textbf{O1}: \textit{Gestione stato ticket}
    \begin{itemize}
      \item \textbf{Descrizione}: l'operatore cambia lo stato del ticket
      \item \textbf{Attori}: operatore
      \item \textbf{Precondizioni}: il ticket deve esistere in database e dev'essere stato assegnato, l'operatore deve aver effettuato il login;
      \item \textbf{Passi principali}: 
        \begin{enumerate}
          \item L'operatore modifica lo stato del ticket
          \item Il sistema aggiorna lo stato del ticket in database
          \item Il sistema notifica l'utente dell'avvenuto cambiamento di stato
        \end{enumerate}
      \item \textbf{Situazioni eccezionali}: nessuna
      \item \textbf{Postcondizioni}: il ticket ha il nuovo stato e l'utente è stato avvisato
    \end{itemize}
  \item \textbf{O2}: \textit{Inserimento di commenti sul ticket}
    \begin{itemize}
      \item \textbf{Descrizione}: un operatore aggiunge commenti su uno specifico ticket
      \item \textbf{Attori}: operatore
      \item \textbf{Precondizioni}: il ticket deve esistere in database e dev'essere stato assegnato, l'operatore deve aver effettuato il login
      \item \textbf{Passi principali}: 
        \begin{enumerate}
          \item L'operatore inserisce un nuovo commento
          \item L'operatore salva il commento
          \item Il sistema inserisce il commento in database
          \item Il sistema notifica l'utente dell'avvenuto inserimento del commento
        \end{enumerate}
      \item \textbf{Situazioni eccezionali}: nessuna
      \item \textbf{Postcondizioni}: il commento è inserito in database
    \end{itemize}
  \item \textbf{A1.1}: \textit{Gestione operatori - inserimento}
    \begin{itemize}
      \item \textbf{Descrizione}: un amministratore inserisce un nuovo operatore
      \item \textbf{Attori}: amministratore
      \item \textbf{Precondizioni}: l'amministratore deve aver effettuato il login
      \item \textbf{Passi principali}: 
        \begin{enumerate}
          \item L'amministratore inserisce i dati del nuovo operatore
          \item L'amministratore salva i nuovi dati
          \item Il sistema inserisce i dati in database
        \end{enumerate}
      \item \textbf{Situazioni eccezionali}: nessuna
      \item \textbf{Postcondizioni}: il nuovo operatore risulta inserito
    \end{itemize}
  \item \textbf{A1.2}: \textit{Gestione operatori - modifica}
    \begin{itemize}
      \item \textbf{Descrizione}: un amministratore modifica un operatore
      \item \textbf{Attori}: amministratore
      \item \textbf{Precondizioni}: l'amministratore deve aver effettuato il login
      \item \textbf{Passi principali}: 
        \begin{enumerate}
          \item L'amministratore modifica i dati dell'operatore
          \item L'amministratore salva i nuovi dati
          \item Il sistema modifica i dati in database
        \end{enumerate}
      \item \textbf{Situazioni eccezionali}: nessuna
      \item \textbf{Postcondizioni}: l'operatore risulta aggiornato
    \end{itemize}
  \item \textbf{A1.3}: \textit{Gestione operatori - elimina}
    \begin{itemize}
      \item \textbf{Descrizione}: un amministratore elimina un operatore
      \item \textbf{Attori}: amministratore
      \item \textbf{Precondizioni}: l'amministratore deve aver effettuato il login
      \item \textbf{Passi principali}: 
        \begin{enumerate}
          \item L'amministratore elimina un operatore
          \item Il sistema aggiorna i dati in database segnando l'operatore come "eliminato"
        \end{enumerate}
      \item \textbf{Situazioni eccezionali}: nessuna
      \item \textbf{Postcondizioni}: l'operatore risulta eliminato
    \end{itemize}
  \item \textbf{A2}: \textit{Assegnamento manuale del ticket}
    \begin{itemize}
      \item \textbf{Descrizione}: un amministratore assegna manualmente il ticket ad un gruppo di operatori
      \item \textbf{Attori}: amministratore
      \item \textbf{Precondizioni}: l'amministratore deve aver effettuato il login
      \item \textbf{Passi principali}: 
        \begin{enumerate}
          \item L'amministratore cambia i dati del ticket inserendo il gruppo a cui assegnarlo
          \item L'amministratore salva i dati
          \item Il sistema modifica i dati in database
        \end{enumerate}
      \item \textbf{Situazioni eccezionali}: nessuna
      \item \textbf{Postcondizioni}: il ticket risulta assegnato al gruppo specificato
    \end{itemize}
  \item \textbf{G1}: \textit{Assegnamento del ticket}
    \begin{itemize}
      \item \textbf{Descrizione}: caso d'uso generico che rappresenta l'assegnamento di un ticket ad un gruppo di operatori
      \item \textbf{Attori}: nessuno
      \item \textbf{Precondizioni}: nessuna
      \item \textbf{Passi principali}: nessuno
      \item \textbf{Situazioni eccezionali}: nessuna
      \item \textbf{Postcondizioni}: il ticket risulta assegnato ad un gruppo di operatori
    \end{itemize}
\end{itemize}

\subsection{Requisiti funzionali}

Di seguito è presente l'elenco completo di tutte le specifiche funzionali necessarie al fine della realizzazione del software:

\begin{enumerate}
  \item \textit{Creazione di un account}, ovvero l'inserimento dei dati relativi per la creazione di un account utente (tra cui nome, password ed email);
  \item \textit{Gestione dell'account}, ovvero la modifica delle informazioni dell'account utente;
  \item \textit{Login} e \textit{logout} di utenti ed operatori;
  \item \textit{Creazione di ticket tramite form} da parte di un utente autenticato;
  \item \textit{Creazione di ticket tramite mail} da parte del sistema;
  \item \textit{Controllo dello stato del ticket}, sia con informazioni aggiuntive (per utenti autenticati) sia senza di esse (per utenti non autenticati);
  \item \textit{Assegnamento automatico del ticket} da parte del sistema ad un gruppo di operatori;
  \item \textit{Assegnamento manuale del ticket} da parte dell'amministratore ad un gruppo di operatori;
  \item \textit{Creazione, modifica ed eliminazione di operatori} da parte dell'amministratore.
\end{enumerate}

Tutte queste funzionalità sono riassunte nella tabella \ref{tab:requisiti-funzionali}.

\begin{table}
  \centering
  \begin{tabular}{|l|l|}
    \hline
    Codice & Descrizione \\
    \hline
    \hline \texttt{FU1} & Creazione account \\
    \hline \texttt{FU2} & Gestione account \\
    \hline \texttt{FU3} & Login e logout \\
    \hline \texttt{FT1} & Creazione ticket tramite form \\
    \hline \texttt{FT2} & Creazione ticket tramite mail \\
    \hline \texttt{FT3} & Controllo stato ticket \\
    \hline \texttt{FT4} & Inserimento commenti \\
    \hline \texttt{FT5} & Gestione stato ticket \\
    \hline \texttt{FT6} & Assegnamento automatico ticket \\
    \hline \texttt{FT7} & Assegnamento manuale ticket \\
    \hline \texttt{FO1} & Creazione operatore \\
    \hline \texttt{FO2} & Modifica operatore \\
    \hline \texttt{FO3} & Eliminazione operatore \\
    \hline
  \end{tabular}
  \caption{Requisiti funzionali}
  \label{tab:requisiti-funzionali}
\end{table}

\newpage
\section{Architettura software}

\subsection{Component Diagram}
\label{sec:component-diagram}

In figura \ref{fig:diag-componenti} è visibile il \textit{component diagram} del sistema.
Questo diagramma rappresenta la struttura interna del sistema dal punto di vista dei componenti principali
e delle relazioni tra di essi.

Come si può vedere, il sistema è suddiviso in due parti:
\begin{itemize}
  \item il \textit{front-end} che si occupa della gestione dell'interfaccia grafica;
  \item il \textit{back-end} che gestisce la \textit{business logic}.
\end{itemize}

Il \textit{back-end} fornisce delle API al \textit{front-end} tramite i vari componenti \textit{Handler} e comunica sia con il DBMS sia
con il server mail esterno.

\begin{figure}[H]
  \centering
  \includegraphics[width=0.85\linewidth]{component-diagram.png}
  \caption{Diagramma dei componenti}
  \label{fig:diag-componenti}
\end{figure}

\subsection{Class Diagram}

Il \textit{class diagram} visibile in figura \ref{fig:diag-class} descrive le varie classi che compongono il sistema.

\begin{figure}[H]
  \centering
  \includegraphics[width=0.95\linewidth]{class-diagram.png}
  \caption{Diagramma delle classi}
  \label{fig:diag-class}
\end{figure}

Le classi \textit{Handler} si occupano della comunicazione con il client e implementano nel codice i vari casi d'uso.
La classe \textit{TicketClassifier} implementa invece l'algoritmo per la classificazione dei ticket ed il loro assegnamento ai gruppi
di operatori.

\section{Architettura hardware}

L'architettura del sistema è una \textit{2-tier architecture}, composta da un client per la GUI e l'interazione con l'utente e da un server
che implementa le logiche che definiscono il sistema.

\subsection{Architettura hardware}

La figura \ref{fig:diag-arch-hw} rappresenta un diagramma architetturale del sistema.

\begin{figure}[H]
  \centering
  \includegraphics[width=0.85\linewidth]{architecture.png}
  \caption{Architettura hardware}
  \label{fig:diag-arch-hw}
\end{figure}

\subsection{Deployment Diagram}

Il \textit{deployment diagram} in figura \ref{fig:diag-deployment} evidenzia i dispositivi considerati e come
i vari componenti identificati alla sezione \ref{sec:component-diagram} sono istanziati su di essi.

\begin{figure}[H]
  \centering
  \includegraphics[width=0.85\linewidth]{deployment-diagram.png}
  \caption{Diagramma di deployment}
  \label{fig:diag-deployment}
\end{figure}

Nel diagramma sono presenti tre \textit{device}:
\begin{enumerate}
  \item Un client che gestirà il \textit{front-end} del sistema;
  \item Un nostro \textit{server} che si occuperà anche del database a fine di semplificare e velocizzare l'interazione tra logica e dati;
  \item Un \textit{server email} esterno che si occuperà delle varie caselle email.
\end{enumerate}
