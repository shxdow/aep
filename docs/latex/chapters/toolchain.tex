Questo capitolo comprende tutte le scelte di librerie, framework e piattaforme che sono state effettuate
per l'implementazione del progetto.

\section{Modellazione}

Per la modellazione sono stati usati i seguenti tool:

\begin{itemize}
  \item \textbf{Diagrammi UML di componenti, classi, architettura e deployment}: \textit{Draw.io}, uno
        strumento sia online sia offline open source per la realizzazione di diagrammi generici, che comprende
        anche strumenti appositi per la modellazione UML;
  \item \textbf{Diagramma dell'implementazione delle classi}: \textit{pyreverse}, un tool in Python in grado di
        estrarre diagrammi rappresentanti le relazioni tra delle classi pre-esistenti in un progetto.
\end{itemize}

\section{Linguaggi e librerie}

\subsection{Back-end}

Per l'implementazione del back-end sono stati usati:

\begin{itemize}
  \item \textbf{Python} come linguaggio di programmazione, scelto per la sua semplicità e portabilità;
  \item \textbf{PIP} come package manager per Python;
  \item \textbf{Django}, un framework Python per la realizzazione di server, gestione di database e templating;
  \item \textbf{SQLite3} come \textit{DBMS}, un dialetto SQL che offre un'ottima integrazione con Python; pur non
        essendo il più performante, è stato scelto per rapidità di implementazione, e in iterazioni future è possibile
        scambiarlo con altri DBMS più performanti;
  \item \textbf{Pylint} per l'\textit{analisi statica} del codice Python, l'identificazione real-time di errori, warning
        e la fornitura di ottimizzazioni e suggerimenti;
  \item \textbf{Django Test Framework} e \textbf{Coverage.py} per l'\textit{analisi dinamica} del codice Python, dei quali
        il primo è integrato in Django, mentre il secondo è uno strumento esterno per misurare la copertura del codice che fornisce
        anche un output interattivo in HTML.
\end{itemize}

\subsection{Front-end}

Per l'implementazione del front-end sono stati usati:

\begin{itemize}
  \item \textbf{JavaScript} come linguaggio di programmazione;
  \item \textbf{HTML e CSS} come linguaggi di markup e styling delle pagine web;
  \item Il framework \textbf{Bootstrap}, che comprende svariati stili di default per velocizzare lo sviluppo di
        una UI coerente e reattiva;
  \item \textbf{React.js} come libreria per la gestione di componenti riusabili ed ottimizzati grazie all'implementazione di
        un \textit{virtual DOM} per il dispatching intelligente di aggiornamenti grafici della pagina web e che offre anche
        la gestione dello stato e dei dati usati dalla Single Page Application;
  \item \textbf{Axios} come libreria JavaScript per effettuare delle richieste AJAX al back-end;
  \item \textbf{Moment}, libreria JavaScript per la gestione di istanti di tempo, giorni, mesi ed intervalli di tempo;
  \item \textbf{Ag-grid} per le tabelle e le griglie, con ottimizzazioni quali la virtualizzazione di righe e colonne,
        la selezione di righe e la ricerca rapida;
  \item \textbf{Lodash}, una libreria JavaScript che fornisce un set di funzioni di utilità base comunemente usate;
  \item \textbf{Jest}, un framework per il \textit{testing} e l'\textit{analisi dinamica} del codice, soprattutto con la libreria React;
  \item \textbf{Testing Library}, una libreria che fornisce alcune utilità per il testing;
  \item \textbf{ESLint}, una libreria per l'\textit{analisi statica} del codice, che fornisce anche suggerimenti, rafforza
        degli standard stilistici e corregge eventuali violazioni di regole custom;
  \item \textbf{Node.js} come JavaScript runtime environment per la compilazione della Single Page Application;
  \item \textbf{NPM} e \textbf{Yarn} come package manager per l'ambiente Node, usati per l'installazione delle varie dipendenze;
  \item \textbf{Electron}, una libreria usata per poter compilare il codice JavaScript, HTML e CSS in una app cross-platform.
\end{itemize}

\section{Infrastruttura}

Per le varie componenti dell'infrastruttura sono stati usati:

\begin{itemize}
      \item \textbf{Travis}, un tool per la \textit{Continuous Integration} ed il testing automatizzato.
\end{itemize}

\section{Documentazione}

La documentazione è stata scritta in \LaTeX, compilata tramite il tool \texttt{latexmk}.

\section{Versioning}

Il sistema di versioning utilizzato è \textbf{Git}, un sistema di versioning distribuito, open source e
gratuito.

Il codice del progetto è presente anche su \textbf{GitHub}, un sito per remote hosting di progetti git.

\section{Ambienti e IDEs}

Lo strumento principale utilizzato è \textbf{Visual Studio Code}, un editor di testo open source e gratuito
realizzato per i sistemi operativi \textit{Linux}, \textit{macOS} e \textit{Windows}.

Esso non è un IDE, in quanto non offre funzionalità di build automation e debugging, ma si basa invece
su un sistema di \textit{estensioni}, sia per syntax highlighting, linting, testing e building, mentre tutti
gli strumenti per il funzionamento dei linguaggi sono locali e forniti dal sistema operativo.

Fra le molte funzionalità, è presente anche la \textit{condivisione live del codice}, in modo da poterne
discutere in real-time e poter usare tecniche di sviluppo agili che richiedono più persone, come il
\textit{pair programming}, anche a distanza.

Come esempio, nella figura \ref{fig:vscode-showcase} è visibile l'editor Visual Studio Code,
che permette anche di gestire diversi progetti, scritti in diversi linguaggi e con diverse tecnologie,
senza dover cambiare editor o IDE.

\begin{figure}[H]
  \centering
  \includegraphics[width=\linewidth]{toolchain/vscode-showcase.jpg}
  \caption{Un esempio dell'uso di Visual Studio Code}
  \label{fig:vscode-showcase}
\end{figure}

È anche stato usato \textbf{VIM}, un editor di testo scritto originariamente per Unix per poter essere
usato da riga di comando, in quei casi in cui era necessario modificare rapidamente un file senza aprire
necessariamente un editor complesso come Visual Studio Code.
Un esempio è visibile in figura \ref{fig:vim-showcase}.

\begin{figure}[H]
  \centering
  \includegraphics[width=\linewidth]{toolchain/vim-showcase.jpg}
  \caption{Esempio di uso di VIM}
  \label{fig:vim-showcase}
\end{figure}
