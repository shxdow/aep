\section{Introduzione}

In questa iterazione vengono implementate le funzionalità base per la gestione dei ticket,
il core della nostra applicazione.

In particolare, le funzionalità implementate sono le seguenti:
\begin{itemize}
  \item \texttt{FT1}, la creazione di ticket tramite form;
  \item \texttt{FT3}, il controllo dello stato del ticket, ma solamente per utenti loggati;
  \item \texttt{FT4}, l'inserimento di commenti sul ticket da parte di utenti loggati ed operatori;
  \item \texttt{FT5}, la gestione dello stato del ticket da parte di operatori.
\end{itemize}

È quindi garantita una comunicazione tra operatori e clienti, funzionalità di base per una
buona \textit{User Experience} in questo ambito.

\section{Back-end}

\subsection{API endpoints}

Nel back-end sono stati realizzati gli endpoint visibili in tabella \ref{tab:endpoint-tickets}.

Questi endpoint sono le fondamenta per il funzionamento base della comunicazione tra operatore e cliente:
\begin{enumerate}
  \item l'utente crea un nuovo ticket tramite l'endpoint \texttt{/v1/ticket/add/};
  \item l'utente visualizza i propri ticket con \texttt{/v1/tickets};
  \item volendo, sia gli utenti sia gli operatori possono visualizzare un ticket con \texttt{/v1/ticket/<pk>}
        ed aggiungere commenti con \texttt{/v1/comment/add/}.
\end{enumerate}

\begin{table}
  \centering
  \begin{tabular}{|l|l|}
    \hline Endpoint & Descrizione \\
    \hline
    \hline \texttt{GET /v1/tickets} & Ritorna l'elenco dei ticket \\
    \hline \texttt{POST /v1/ticket/add/} & Crea un nuovo ticket \\
    \hline \texttt{GET /v1/ticket/<pk>/} & Ottiene le informazioni di un ticket \\
    \hline \texttt{PUT /v1/ticket/<pk>/} & Aggiorna lo stato di un ticket \\
    \hline \texttt{POST /v1/comment/add/} & Crea un nuovo commento su un ticket \\
    \hline
  \end{tabular}
  \caption{Endpoint}
  \label{tab:endpoint-tickets}
\end{table}

Di seguito un estratto di codice che rappresenta le varie funzioni per la creazione di ticket
e di commenti.

\begin{lstlisting}[language=Python]
@login_required
@api_view(['POST'])
def add_ticket(request):
    try:
        ticket = Ticket.objects.create(
          title=request.data["title"],
          description=request.data["description"],
          client=Client.objects.get(pk=request.data["client"]))

        ticket.save()
    except Exception as e:
        return Response(status=status.HTTP_500_INTERNAL_SERVER_ERROR)
    return Response(status=status.HTTP_201_CREATED)


@login_required
@api_view(['POST'])
def add_comment(request):
    try:
        account = Account.objects.get(pk=request.data["account"])
        ticket = Ticket.objects.get(pk=request.data["ticket"])
        comment = Comment.objects.create(
          timestamp=request.data["timestamp"],
          ticket=ticket,
          account=account,
          content=request.data["content"])

        comment.save()
    except Exception as e:
        return Response(status=status.HTTP_500_INTERNAL_SERVER_ERROR)
    return Response(status=status.HTTP_201_CREATED)
\end{lstlisting}

\subsection{Testing ed analisi statica}

Alla fine di questa iterazione la qualità del codice è di $9.23$ su $10$ secondo \textit{pylint}.

La coverage invece è visibile nella relativa cartella ed ammonta al $x \%$, con $25$ test positivi.

\section{Front-end}

\subsection{Pagine aggiunte}

Il front end rispecchia le nuove API aggiunte. In particolare, sono state realizzate le seguenti pagine:

\begin{itemize}
  \item lista dei ticket;
  \item creazione di un ticket, solo per gli utenti loggati;
  \item visualizzazione delle informazioni di un ticket, per operatori ed utenti loggati.
\end{itemize}

Queste tre pagine racchiudono l'intera UX e UI che permettono ad un utente di sapere quali ticket ha aperto,
in che stato si trovano e come gli operatori stanno gestendo il problema.

Nella figura \ref{fig:front-end-ticket-fsm} sono visibili queste tre pagine e l'utente può navigare da una all'altra.
È da ricordare che, su browser, in qualunque momento è possibile ritornare alla pagina precedente grazie
alla navigazione integrata.

\begin{figure}
  \centering
  \includegraphics[width=0.9\linewidth]{iterazione2/front-end-ticket-fsm.png}
  \caption{Relazioni tra le pagine dei ticket}
  \label{fig:front-end-ticket-fsm}
\end{figure}

Di seguito sono presenti varie immagini che rappresentano le pagine implementate su front-end.

\begin{figure}[H]
  \centering
  \includegraphics[width=\linewidth]{iterazione2/ticket-list-page.jpg}
  \caption{Pagina di elenco ticket}
  \label{fig:ticket-list-page}
\end{figure}

\begin{figure}[H]
  \centering
  \includegraphics[width=\linewidth]{iterazione2/ticket-new-page.jpg}
  \caption{Pagina di creazione ticket}
  \label{fig:ticket-new-page}
\end{figure}

\begin{figure}[H]
  \centering
  \includegraphics[width=\linewidth]{iterazione2/ticket-info-page.jpg}
  \caption{Le informazioni del ticket}
  \label{fig:ticket-info-page}
\end{figure}

\begin{figure}[H]
  \centering
  \includegraphics[width=\linewidth]{iterazione2/ticket-change-status-page.jpg}
  \caption{Un operatore cambia lo stato del ticket}
  \label{fig:ticket-change-status}
\end{figure}

\subsection{Testing, coverage ed analisi statica}

Alla fine di questa iterazione, \textit{ESLint} non ha trovato alcun problema nel codice.

Come visibile in figura \ref{fig:front-end-tests-it2}, i test del front-end consistono in questo istante
dello sviluppo:
\begin{enumerate}
  \item $10$ suite di test;
  \item $52$ casi di test.
\end{enumerate}

\begin{figure}[H]
  \centering
  \includegraphics[width=\linewidth]{iterazione2/front-end-tests-it2.jpg}
  \caption{Unit testing front-end}
  \label{fig:front-end-tests-it2}
\end{figure}

La coverage ammonta ai risultati visibili nella tabella \ref{tab:code-coverage-front-end-it2}.

\begin{table}
  \centering
  \begin{tabular}{|l|r|}
    \hline Metrica & Copertura percentuale \\
    \hline
    \hline Istruzioni & $84.07\%$ \\
    \hline Branches   & $77.24\%$ \\
    \hline Funzioni   & $82.14\%$ \\
    \hline Righe      & $84.92\%$ \\
    \hline
  \end{tabular}
  \caption{Coverage totale per il front-end}
  \label{tab:code-coverage-front-end-it2}
\end{table}